
\documentclass[a4paper,12pt]{article}
\usepackage{enumitem} % -> Alphabetical Lists
\usepackage{amsmath} % -> Matrices
\usepackage{fullpage} % -> A4 Full Page
\usepackage{amssymb} % -> Therefore
\usepackage[utf8]{inputenc}
\usepackage{graphicx}
\usepackage{adjustbox}
\usepackage{listings}
\usepackage{braket}
\usepackage{geometry}
\usepackage{tikz}
\usetikzlibrary{quantikz}

\graphicspath{ {./} }

\geometry{
    a4paper,
    total={170mm,257mm},
    left=20mm,
    top=20mm,
}

\title{Quantum Computing Assignment 2 - Group 18}
\author{
    Rallabhandi, Anand Krishna 
    \and
    Mustafa, Syed Husain
    \and
     , Mohammed Kamran 
}
\date{\today}

\begin{document}

\maketitle

\section*{Exercise 2.1}

 (Basic quantum circuits)

\begin{enumerate}[label=(\alph*)]
    \item
          For a 2 qubit state $\ket{\phi}$ the following are valid transitions for a Swap Gate:
          \[ \ket{0,0} \mapsto \ket{0,0}\]
          \[ \ket{0,1} \mapsto \ket{1,0}\]
          \[ \ket{1,0} \mapsto \ket{0,1}\]
          \[ \ket{1,1} \mapsto \ket{1,1}\]
          For Unitary Matrix $U \epsilon \hspace{2mm}\mathbb{C}^{2n \times 2n}$, where $U(\ket{a,b}) \mapsto (\ket{b,a}$ implies U is given by
          \[\begin{pmatrix}
                  1 & 0 & 0 & 0 \\
                  0 & 0 & 1 & 0 \\
                  0 & 1 & 0 & 0 \\
                  0 & 0 & 0 & 1 \\
              \end{pmatrix}_{4 \times 4}
          \]
          It can be shown that the Swap Gate is equivalent to 3 Controlled NOT Gates,\\
          \[\ket{a,b} \mapsto \ket{a, a\oplus b}, \hspace{2mm}here\hspace{2mm} a\hspace{2mm} is\hspace{2mm} control\hspace{2mm} and \hspace{2mm}b\hspace{2mm} is\hspace{2mm} target\]
          \[\ket{a\oplus (a\oplus b), a\oplus b}, \hspace{2mm}here\hspace{2mm} a\oplus b\hspace{2mm} is\hspace{2mm} control\hspace{2mm} and \hspace{2mm}a\hspace{2mm} is\hspace{2mm} target\]
          \[\ket{b, (a\oplus b)\oplus b},\hspace{2mm}{here\hspace{2mm} b\hspace{2mm} is\hspace{2mm} control\hspace{2mm} and \hspace{2mm}a\oplus b\hspace{2mm} is\hspace{2mm} target}\]
          \[\ket{b, a}, \textup{\hspace{2mm}the\hspace{2mm}result\hspace{2mm}is\hspace{2mm}the\hspace{2mm}swap\hspace{2mm}of\hspace{2mm}}\ket{a,b}\]
          Therefore the circuits are equivalent
          %   \begin{quantikz}
          %       & \swap{2} & \qw \\
          %       & targX{} &  \qw
          %   \end{quantikz}
          \begin{quantikz}[column sep=0.3cm]
              &\swap{1} & \qw\\
              &\targX{} & \qw
          \end{quantikz}\hspace{4mm}=
          \begin{quantikz}[column sep=0.3cm]
              & \ctrl{1}  & \targ{} & \ctrl{1} & \qw \\
              &\targ{} & \ctrl{-1} & \targ{} & \qw
          \end{quantikz}

          %   \lstick{$\ket{0}$}

    \item
          \[\begin{quantikz}[column sep = 0.5cm]
                  \lstick{$\ket{0}$}& \gate{H} & \ctrl{1} &\qw \\
                  \lstick{$\ket{0}$}& \qw      &\targ{}   &\qw
              \end{quantikz} \vspace{2mm} \Bigg\} \ket{\psi} \]
          \linebreak
          The Hadamard Gate transforms the top qubit into a superposition state, this then acts as a control input to the CNOT Gate. The target get's inverted only when control is 1.
          In general the outputs of the given circuit, ie; Hadamard Gate followed by CNOT with a 2 qubit input, are known as \textbf{Bell States}. They are given by
          \[\ket{\beta}_{xy} = \frac{\ket{0, y}+(-1)^{x}\ket{1,\overline{y}}}{\sqrt{2}}\]

          Thus for $\ket{00}$, \[\beta_{00} = \frac{\ket{0, 0}+(-1)^{0}\ket{1,\overline{0}}}{\sqrt{2}} = \frac{1}{\sqrt{2}}\Big(\ket{00}+ \ket{11}\Big)\]

          Alternatively, this can be shown as follows:
          \[H\ket{0} = \frac{1}{\sqrt{2}}\begin{pmatrix}1 & 1  \\
                  1 & -1\end{pmatrix}_{2 \times 2} \begin{pmatrix}1 \\
                  0\end{pmatrix}_{2 \times 1} = \frac{1}{\sqrt{2}}\begin{pmatrix}1 \\
                  1\end{pmatrix}_{2 \times 1} = \frac{1}{\sqrt{2}}( {\ket{0} + \ket{1}})\]


          \[\Big(\frac{1}{\sqrt{2}}(\ket{0}+\ket{1})\Big)\ket{0}\xrightarrow{\text{CNOT}}\frac{1}{\sqrt{2}}\Big(\ket{00}+ \ket{11}\Big)\]

    \item
          We know that \[\ket{0} \xrightarrow{\text{H}} \frac{1}{\sqrt{2}}\Big(\ket{0}+\ket{1}\Big) \hspace{3mm}and\hspace{3mm}\ket{1} \xrightarrow{\text{H}} \frac{1}{\sqrt{2}}\Big(\ket{0}-\ket{1}\Big) \hspace{10mm} \textbf{(A)}\]
          For the Pauli-Z Unitary Operator \[\ket{0}\xrightarrow{\text{Z}}\ket{0}\hspace{3mm} and\hspace{3mm} \ket{1}\xrightarrow{\text{Z}}-\ket{1} \hspace{51mm} \textbf{(B)}\] \\
          Using the results from \textbf{(A)} \& \textbf{(B)} \\ \linebreak
          Consider $\ket{\psi} = \ket{10}$ where 1 is the control qubit and 0 is the target qubit, our desired output is $\ket{\psi} = \ket{11}$.
          We perform the following operations on the target qubit with Controlled-Z Gate switched on for control set to 1. \\ \linebreak
          \[\ket{0} \xrightarrow{\text{H}} \frac{1}{\sqrt{2}}\Big(\ket{0}+\ket{1}\Big) \xrightarrow{\text{Z}}\frac{1}{\sqrt{2}}\Big(\ket{0}-\ket{1}\Big) \xrightarrow{\text{H}} \frac{1}{\sqrt{2}} \Bigg(\Big(\frac{1}{\sqrt{2}}\Big(\ket{0}+\ket{1}\Big) - \Big(\frac{1}{\sqrt{2}}\Big(\ket{0}-\ket{1}\Big)\Bigg) \xrightarrow{} \ket{1}\] \\ \linebreak
          The above operations can be realized in the below Quantum Circuit \\ \linebreak
          \[\begin{quantikz}
                  \lstick{$\ket{0}$}& \gate{H} & \gate{Z} & \gate{H} &\qw \\
                  \lstick{$\ket{1}$} & \qw      &\ctrl{-1}   &\qw &\qw
              \end{quantikz}\]
          \\ \linebreak

\end{enumerate}
\section*{Exercise 2.2}

 (IBM Q and Qiskit)

\begin{enumerate}[label=(\alph*)]
    \item
          \begin{minipage}[t]{\linewidth}
              \includegraphics[scale=0.5]{circuit}
          \end{minipage}
          \begin{minipage}[t]{\linewidth}
              \includegraphics[scale=0.5]{histogram}
          \end{minipage}
    \item
          IBM Quantum Lab Jupyter Notebook Code
          \begin{lstlisting}[language=Python]
%matplotlib inline
# Importing standard Qiskit libraries and configuring account
from qiskit import QuantumCircuit, execute, Aer, IBMQ
from qiskit.compiler import transpile, assemble
from qiskit.tools.jupyter import *
from qiskit.visualization import *

from qiskit import QuantumRegister, ClassicalRegister, QuantumCircuit
from numpy import pi

# Loading your IBM Q account(s)
provider = IBMQ.load_account()
        \end{lstlisting}
          \begin{lstlisting}
qreg_q = QuantumRegister(2, 'q')
creg_c = ClassicalRegister(2, 'c')
circuit = QuantumCircuit(qreg_q, creg_c)

circuit.reset(qreg_q[0])
circuit.reset(qreg_q[1])
circuit.h(qreg_q[0])
circuit.cx(qreg_q[0], qreg_q[1])
circuit.measure(qreg_q[0], creg_c[0])
circuit.measure(qreg_q[1], creg_c[1])

circuit.draw()
        \end{lstlisting}
          \begin{minipage}[t]{\linewidth}
              \includegraphics[scale=0.5]{JupyterFigure}
          \end{minipage}
          \begin{lstlisting}
simulator = Aer.get_backend('qasm_simulator')

result = execute(circuit, simulator).result()
counts = result.get_counts(circuit)
plot_histogram(counts, title='Results')
        \end{lstlisting}
          \begin{minipage}[t]{\linewidth}
              \includegraphics[scale=0.5]{JupyterHistogram}
          \end{minipage}

\end{enumerate}

\end{document}
