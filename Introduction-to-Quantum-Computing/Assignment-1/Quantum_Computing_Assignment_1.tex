\documentclass[a4paper,12pt]{article}
\usepackage[spanish,es-lcroman]{babel}
\usepackage[utf8]{inputenc}
\usepackage[scale=0.75]{geometry}
\usepackage{braket}
\usepackage{listings}
\usepackage{color}
\usepackage{enumitem} % -> Alphabetical Lists
\usepackage{amsmath} % -> Matrices
\usepackage{fullpage} % -> A4 Full Page
\usepackage{amssymb} % -> Therefore
\usepackage{tabstackengine}
\usepackage{amsfonts}

\title{Quantum Computing Assignment 1 - Group 18}
\author{
    Rallabhandi, Anand Krishna 
    \and
    Mustafa, Syed Husain
    \and
     , Mohammed Kamran 
}
\date{9th November 2020}

\begin{document}


\maketitle

\section*{Exercise 1.1}

 (Bloch sphere and single qubit quantum gates)

\begin{enumerate}[label=(\alph*)]
    \item $|\psi\rangle = e^{i\gamma}[ \cos(\frac{\theta}{2})|o\rangle +       e^{i\phi}\sin(\frac{\theta}{2})|1\rangle ] $
          \\~\\
          $|\psi\rangle =i[\frac{1}{2}|0\rangle + \frac{\sqrt{3}}{2}i|1\rangle ]$
          \\~\\
          $e^{i\gamma} = i$ This implies $\gamma = \frac{\pi}{2}$
          \\~\\
          $\cos(\frac{\theta}{2}) = \frac{1}{2}$ This implies $\theta = \frac{2\pi}{3}$
          \\~\\
          $e^{i\phi}\sin(\frac{\theta}{2}) = \frac{\sqrt{3}}{2}i$ This implies $\phi = \frac{\pi}{2}$

    \item \vspace{5mm}
          \begin{flushleft}
              Rotation operation about Pauli X Matrix is given by
          \end{flushleft}

          \[R_{X}(\theta) := e^{\frac{-i\theta X}{2}} = \cos{\frac{\theta}{2}}{I} - i\sin{\frac{\theta}{2}}X\]
          Based on the above
          \[R_{x}(\frac{2\pi}{3}) = \cos{\frac{\pi}{3}}I - i\sin{\frac{\pi}{3}}X = \setstackgap{L}{1.1\baselineskip}
              \fixTABwidth{T}
              \parenMatrixstack{
                  \frac{1}{2}\hspace{2mm} & 0\\
                  0\hspace{2mm} & \frac{1}{2}
              }_{ 2 \times 2} -\hspace{3mm} \frac{\sqrt{3}i}{2}
              \setstackgap{L}{1.1\baselineskip}
              \fixTABwidth{T}
              \parenMatrixstack{
                  0\hspace{2mm} & 1\\
                  1\hspace{2mm} & 0
              }_{2 \times 2}  = \begin{pmatrix}
                  \frac{1}{2}\hspace{2mm}         & -\frac{\sqrt{3}}{2}i \\
                  -\frac{\sqrt{3}}{2}\hspace{2mm} & \frac{1}{2}
              \end{pmatrix}_{2 \times 2} \hspace{3mm} \textbf{(A)}\]
          \vspace{5mm}
          \begin{flushleft}
              Given $\ket{\phi} = \begin{pmatrix}
                      \frac{1}{2} \\
                      \frac{\sqrt{3}}{2}i
                  \end{pmatrix}_{2 \times 1} \hspace{5mm} \textbf{(B)}$
          \end{flushleft}

          \vspace{3mm} \textbf{(A)} times \textbf{(B)} gives, $\begin{pmatrix}
                  \frac{1}{2}\hspace{2mm}         & -\frac{\sqrt{3}}{2}i \\
                  -\frac{\sqrt{3}}{2}\hspace{2mm} & \frac{1}{2}
              \end{pmatrix}_{2 \times 2} \begin{pmatrix}
                  \frac{1}{2} \\
                  \frac{\sqrt{3}}{2}i
              \end{pmatrix}_{2 \times 1}  = \frac{1}{4} \begin{pmatrix}
                  1 \hspace{2mm}          & 3         \\
                  -\sqrt{3}i \hspace{2mm} & \sqrt{3}i
              \end{pmatrix}_{2 \times 1}$
          \vspace{5mm}

    \item
          Since the Hadamard gate ($H$) is a Unitary Matrix
          It can be represented as follows

          $U = e^{i\alpha}R_{x}(\beta)R_{y}(\gamma)R_{z}(\delta)$

          Simplifying the above equations through Matrix Multiplication we get the following

          $U = \begin{pmatrix}
                  e^{i(\alpha - \beta / 2 - \delta /  2)}\cos{\gamma / 2} & -e^{i(\alpha - \beta / 2 + \delta /  2)}\sin{\gamma / 2} \\
                  e^{i(\alpha + \beta / 2 - \delta /  2)}\sin{\gamma / 2} & e^{i(\alpha + \beta / 2 + \delta /  2)}\cos{\gamma / 2}
              \end{pmatrix}$

          The Hadamard gate is defined as

          $H = \frac{1}{\sqrt{2}}
              \begin{pmatrix}
                  1 & 1  \\
                  1 & -1
              \end{pmatrix}$

          Since \emph{H} and \emph{U} represent the same matrix, we equate element wise and arrive at

          $e^{i(\alpha - \beta / 2 - \delta /  2)}\cos{\gamma / 2} = \frac{1}{\sqrt{2}}$ \hfill(1)\\
          $-e^{i(\alpha - \beta / 2 + \delta /  2)}\sin{\gamma / 2} = \frac{1}{\sqrt{2}}$ \hfill(2)\\
          $e^{i(\alpha + \beta / 2 - \delta /  2)}\sin{\gamma / 2} = \frac{1}{\sqrt{2}}$ \hfill(3)\\
          $e^{i(\alpha + \beta / 2 + \delta /  2)}\cos{\gamma / 2} = -\frac{1}{\sqrt{2}}$ \hfill(4)

          From the above equations it can be inferred that $\gamma = \pi / 2$

          Simplifying (1)-(4) we arrive at

          $e^{i(\alpha - \beta / 2 - \delta /  2)} = 1$ \hfill(5)\\
          $e^{i(\alpha - \beta / 2 + \delta /  2)} = -1$ \hfill(6)\\
          $e^{i(\alpha + \beta / 2 - \delta /  2)} = 1$ \hfill(7)\\
          $e^{i(\alpha + \beta / 2 + \delta /  2)} = -1$ \hfill(8)

          From (5),(7) using $e^{i0} = 1$

          $\alpha - \beta / 2 - \delta /  2 = 0$\\
          $\alpha + \beta / 2 - \delta /  2 = 0$

          From (6),(8) using $e^{i\pi} = -1$


          $\alpha - \beta / 2 + \delta /  2 = \pi$\\
          $\alpha + \beta / 2 + \delta /  2 = \pi$


          On solving the above equations we get

          $\alpha = \pi / 2, \beta = 0, \delta = \pi$

          $\therefore H = e^{i\pi / 2}R_{x}(0)R_{y}(\pi / 2)R_{z}(\pi)$


\end{enumerate}

\section*{Exercise 1.2}

 (Multiple qubits and tensor products)

\begin{enumerate}[label=(\alph*)]
    \item
          Y =
          $\begin{pmatrix}
                  0 & -i \\
                  i & 0
              \end{pmatrix}$
          Z =
          $\begin{pmatrix}
                  1 & 0  \\
                  0 & -1
              \end{pmatrix}$
          %   \begin{align*}

          %   \end{align*}

          Y \otimes Z
          = \begin{pmatrix}
              y_{11}Z & y_{12}Z \\
              y_{21}Z & y_{22}Z
          \end{pmatrix}


          &=
          \begin{pmatrix}
              y_{11} $\begin{pmatrix}
                      1 & 0  \\
                      0 & -1
                  \end{pmatrix}$    & y_{12}    $\begin{pmatrix}
                      1 & 0  \\
                      0 & -1
                  \end{pmatrix}$ \\
              y_{21}    $\begin{pmatrix}
                      1 & 0  \\
                      0 & -1
                  \end{pmatrix}$ & y_{22}    $\begin{pmatrix}
                      1 & 0  \\
                      0 & -1
                  \end{pmatrix}$
          \end{pmatrix}\\
          %\linebreak
          &=
          \begin{pmatrix}
              0 $\begin{pmatrix}
                      1 & 0  \\
                      0 & -1
                  \end{pmatrix}$    & -i    $\begin{pmatrix}
                      1 & 0  \\
                      0 & -1
                  \end{pmatrix}$ \\
              i    $\begin{pmatrix}
                      1 & 0  \\
                      0 & -1
                  \end{pmatrix}$ & 0    $\begin{pmatrix}
                      1 & 0  \\
                      0 & -1
                  \end{pmatrix}$
          \end{pmatrix}\\
          %\linebreak
          &=
          \begin{pmatrix}
              0 & 0  & -i & 0 \\
              0 & 0  & 0  & i \\
              i & 0  & 0  & 0 \\
              0 & -i & 0  & 0 \\
          \end{pmatrix}\\
          \end{align*}

    \item (b) Code to Compute Kronecker Product of Pauli Y and Z

          \begin{lstlisting}
        import numpy as np
        
        # Pauli Y
        Y = np.array([[0, -np.complex(0,1)], [np.complex(0,1), 0]])
        # Pauli Z
        Z = np.array([[1, 0], [0, -1]])
        
        # Computing the Kronecker product
        kron_YZ = np.kron(Y, Z)
        print(kron_YZ)
        \end{lstlisting}
          \hspace{10mm}Output
          \begin{lstlisting}
        [[ 0.+0.j  0.+0.j  0.-1.j  0.-0.j]
         [ 0.+0.j -0.+0.j  0.-0.j  0.+1.j]
         [ 0.+1.j  0.+0.j  0.+0.j  0.+0.j]
         [ 0.+0.j -0.-1.j  0.+0.j -0.+0.j]]
        \end{lstlisting}

          % \item To verify (Y \otimes Z)(
    \item
          \vspace{5mm}
          \begin{flushleft}
              (c) \hspace{3mm}Evaluating $ (Y \otimes Z)(\ket{v} \otimes \ket{w})$
          \end{flushleft}

          \[
              Given, \hspace{5mm} v = \begin{pmatrix}
                  \frac{3}{5} \\
                  \frac{4}{5}
              \end{pmatrix}_{2 \times 1} \hspace{5mm} and \hspace{5mm} \text{w = } \begin{pmatrix}
                  0 \\
                  1
              \end{pmatrix}_{2 \times 1}\]
          \vspace{5mm} Performing Kronecker Product according to Eqn(3),
          \[
              (v \otimes \hspace{0.5mm} w) = \begin{pmatrix}
                  \frac{3}{5} \\
                  \frac{4}{5}
              \end{pmatrix}_{2 \times 1} \otimes \hspace{4mm} \begin{pmatrix}
                  0 \\
                  1
              \end{pmatrix}_{2 \times 1}  =  \begin{pmatrix}
                  0           \\
                  \frac{3}{5} \\
                  0           \\
                  \frac{4}{5}
              \end{pmatrix}_{4 \times 1}
          \]

          From Q1(a) we know,
          \vspace{5mm}
          \[
              \setstackgap{L}{1.1\baselineskip}
              \fixTABwidth{T}
              (Y \otimes \hspace{0.5mm} Z) = \parenMatrixstack{
                  0 & 0 & -i & 0\\
                  0 & 0 & 0 & i \\
                  i & 0 & 0 & 0 \\
                  0 & -i & 0 & 0
              }_{ 4 \times 4}
          \]

          \vspace{5mm}
          \[
              (Y \otimes \hspace{0.5mm} Z)(v \otimes \hspace{0.5mm} w) =
              \setstackgap{L}{1.1\baselineskip}
              \fixTABwidth{T}
              \parenMatrixstack{
                  0 & 0 & -i & 0\\
                  0 & 0 & 0 & i \\
                  i & 0 & 0 & 0 \\
                  0 & -i & 0 & 0
              }_{ 4 \times 4} \hspace{3mm} \begin{pmatrix}
                  0           \\
                  \frac{3}{5} \\
                  0           \\
                  \frac{4}{5}
              \end{pmatrix}_{2 \times 1} = \begin{pmatrix}
                  0            \\
                  \frac{4}{5}i \\
                  0            \\
                  -\frac{3}{5}i
              \end{pmatrix}_{4 \times 1}
              \hspace{7mm} \textbf{(A)}
          \]

          Evaluating $(Y \ket{v}) \otimes (Z \ket{w})$

          \vspace{5mm}

          \[
              Y\ket{v} = \setstackgap{L}{1.1\baselineskip}
              \fixTABwidth{T}
              \parenMatrixstack{
                  0 & -i\\
                  i & 0
              }_{2 \times 2} \begin{pmatrix}
                  \frac{3}{5} \\
                  \frac{4}{5}
              \end{pmatrix}_{2 \times 1} =     \begin{pmatrix}
                  -\frac{4}{5}i \\
                  \frac{3}{5}i
              \end{pmatrix}_{2 \times 1}
          \]
          \vspace{5mm}
          \[
              Z\ket{w} = \setstackgap{L}{1.1\baselineskip}
              \fixTABwidth{T}
              \parenMatrixstack{
                  1 & 0\\
                  0 & -1
              }_{2 \times 2} \begin{pmatrix}
                  0 \\
                  1
              \end{pmatrix}_{2 \times 1} =     \begin{pmatrix}
                  0 \\
                  -1
              \end{pmatrix}_{2 \times 1}
          \]
          \vspace{5mm}
          \[ (Y \ket{v}) \otimes (Z \ket{w}) =  \begin{pmatrix}
                  -\frac{4}{5}i \\
                  \frac{3}{5}i
              \end{pmatrix}_{2 \times 1} \otimes \hspace{4mm} \begin{pmatrix}
                  0 \\
                  -1
              \end{pmatrix}_{2 \times 1} = \begin{pmatrix}
                  0            \\
                  \frac{4}{5}i \\
                  0            \\
                  -\frac{3}{5}i
              \end{pmatrix}_{4 \times 1}
              \hspace{7mm} \textbf{(B)}\]


          \vspace{0.5cm}

          \begin{center}
              \emph{As \textbf{(A)} \& \textbf{(B)} are equal. Hence the results are in agreement.}
          \end{center}

    \item
          \vspace{10mm}\begin{flushleft}
              (d)\hspace{3mm}Let $ |v\rangle = \begin{pmatrix} a \\ b\end{pmatrix}$ and $ |w\rangle = \begin{pmatrix} c \\ d\end{pmatrix}$
              \\~\\
              $|v\rangle \otimes |w\rangle = ( ac , ad , bc , bd )^T $
              \\~\\
              $ ac=1/\sqrt{2}, bd=1, ad=0, bc=1/\sqrt{2} $
              \\~\\
              On solving we get $a/b = 0 , ac=1/\sqrt{2}$
              \\~\\
              This implies a=0. But this is not possible since $ac=1/\sqrt{2}$.
              \\~\\
              Thus there does not exist any value of a,b,c,d
          \end{flushleft}

\end{enumerate}

\end{document}