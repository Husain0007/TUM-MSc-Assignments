\documentclass[a4paper, 12pt]{article}
\usepackage[utf8]{inputenc}
\usepackage{amsmath,amsfonts,amssymb,fixltx2e}
\usepackage{fullpage}
\usepackage{braket}
\usepackage{tikz}
\usetikzlibrary{quantikz}

\title{\fontfamily{lmss}\selectfont{Introduction to Quantum Computing \linebreak Assignment 3 - Group 18\\
}}
\author{
    Rallabhandi, Anand Krishna 
    \and
    Mustafa, Syed Husain
    \and
     , Mohammed Kamran 
}
\date{November 2020}

\begin{document}

\maketitle

\section*{Ex 3.1 Basis Transformation and Measurement}
    \section*{\small(a)}


\hspace{13mm}We need to show that,
\begin{equation}
|\SI{\pm}\rangle = \frac{1}{\sqrt{2}}(|0\rangle \SI{\pm} |1\rangle) 
\end{equation}
\hspace{20mm}is orthogonal, this implies we need to prove that \\

\begin{gather*}
\langle  + | + \rangle = 1, \langle  - | - \rangle = 1,  \langle + | -\rangle = 0 \\~\\
\langle  + | + \rangle = \langle + \rangle ^\intercal \langle + \rangle
= \frac{1}{\sqrt{2}}(|0\rangle + |1\rangle)^\intercal * \frac{1}{\sqrt{2}}(|0\rangle + |1\rangle)
= \frac{1}{2}\begin{bmatrix}
1 &  1\\
\end{bmatrix}
\begin{bmatrix}
1\\
1
\end{bmatrix}
= 1
\\~\\
\langle  - | - \rangle = \langle - \rangle ^\intercal \langle - \rangle
= \frac{1}{\sqrt{2}}(|0\rangle + |1\rangle)^\intercal * \frac{1}{\sqrt{2}}(|0\rangle - |1\rangle)
= \frac{1}{2}\begin{bmatrix}
1 & -1\\
\end{bmatrix}
\begin{bmatrix}
1\\
-1
\end{bmatrix}
= 1
\\~\\
\langle  + | - \rangle = \langle + \rangle ^\intercal \langle - \rangle
= \frac{1}{\sqrt{2}}(|0\rangle + |1\rangle)^\intercal * \frac{1}{\sqrt{2}}(|0\rangle - |1\rangle)
= \frac{1}{2}\begin{bmatrix}
1 & 1\\
\end{bmatrix}
\begin{bmatrix}
1\\
-1
\end{bmatrix}
= 0
\\ 
\end{gather*}
\\~\\
\section*{\small(b)}
\begin{gather*}
    |\psi\rangle = \frac{i}{\sqrt{2}}|0\rangle + \frac{1}{\sqrt{2}}|1\rangle\\~\\
    |U_1\rangle = \frac{3}{5}|0\rangle + \frac{4i}{5}|1\rangle\\~\\
    |U_2\rangle = \frac{4}{5}|0\rangle - \frac{3i}{5}|1\rangle\\
    \frac{5}{3} * |U_1\rangle + \frac{4}{5} * |U_2\rangle = \frac{25}{12}i|1\rangle, \hspace{0.5cm}i|1\rangle = \frac{4}{5}|U_1\rangle - \frac{3}{5}|U_2\rangle\\~\\
    \frac{5}{4} * |U_1\rangle + \frac{5}{3} * |U_2\rangle = \frac{25}{12}|0\rangle,  \hspace{0.5cm}|0\rangle = \frac{3}{5}|U_1\rangle + \frac{4}{5}|U_2\rangle \\~\\
|\psi\rangle = -(\frac{i}{5\sqrt{2}})|U_1\rangle +  (\frac{7i}{5\sqrt{2}})|U_2\rangle\\~\\
P_1 = |\frac{i}{5\sqrt{2}}^2| = \frac{1}{50}, \hspace{0.5cm}
P_2 = |\frac{7i}{5\sqrt{2}}|^2 = \frac{49}{50}
\end{gather*}
\section*{\small(c)}
\[CNOT-INV = \begin{bmatrix}
1 & 0 & 0 & 0 \\
0 & 0 & 0 & 1 \\
0 & 0 & 1 & 0 \\
0 & 1 & 0 & 0 
\end{bmatrix}\]
\\
\hspace{20mm}\vspace{3mm}The left hand side of the diagram equation is $(H \otimes H)( CNOT-INV )(H \otimes H)$
\begin{gather*}
\hspace{20mm}(H \otimes H)( CNOT-INV )(H \otimes H) \linebreak
    
    \[ =\frac{1}{4}\begin{bmatrix}
1 & 1 & 1 & 1 \\
1 & -1 & 1 & -1 \\
1 & 1 & -1 & -1 \\
1 & -1 & -1 & 1 
\end{bmatrix}   
\begin{bmatrix}
1 & 0 & 0 & 0 \\
0 & 0 & 0 & 1 \\
0 & 0 & 1 & 0 \\
0 & 1 & 0 & 0 
\end{bmatrix}
\begin{bmatrix}
1 & 1 & 1 & 1 \\
1 & -1 & 1 & -1 \\
1 & 1 & -1 & -1 \\
1 & -1 & -1 & 1 
\end{bmatrix}\\\]
\[=\frac{1}{4}\begin{bmatrix}
4 & 0 & 0 & 0 \\
0 & 4 & 0 & 0 \\
0 & 0 & 0 & 4 \\
0 & 0 & 4 & 0
\end{bmatrix}  
= CNOT\]
\hspace{20mm}Hence the following circuits are equivalent
\[
    \begin{quantikz}[column sep=0.3cm]
        &\gate{H} & \targ{}  & \gate{H}\qw\\
        &\gate{H} & \ctrl{-1} & \gate{H} &\qw
    \end{quantikz}\hspace{4mm}=
    \begin{quantikz}[column sep=0.3cm]
        & \ctrl{1} & \qw \\
        &\targ{} & \qw
    \end{quantikz}
\]
\vspace{5mm}
\hspace{18mm}Using the equivalent CNOT circuit to derive the following relations \\

\[\begin{quantikz}
    & \gate{H}\slice{$\ket{\psi_{1}}$} &\qw & \targ{}\slice{$\ket{\psi_{2}}$} &\qw & \gate{H}\slice{$\ket{\psi_{3}}$} & \qw \\
    & \gate{H} &\qw & \ctrl{-1} &\qw & \gate{H} & \qw 
    \end{quantikz}
\]
For $\ket{+}\ket{+}\xrightarrow{\text{CNOT}}\ket{+}\ket{+}$
\[\ket{\psi_{1}} = (H\ket{+}) \otimes (H\ket{+}) \xrightarrow{\text{}}\ket{0} \ket{0} \hspace{10mm}(\hspace{2mm}From \hspace{2mm}Hint\hspace{2mm})\]
\[\hspace{5mm}\ket{\psi_{2}} =(CNOT-INV)(\ket{\psi_{1}}) \xrightarrow{\text{}} {\text{}}\ket{0}\ket{0} \hspace{5mm}(\hspace{2mm}Target \hspace{2mm}Qubit \hspace{2mm} is\hspace{2mm}0\hspace{2mm})\]
\[\ket{\psi_{3}} = (H\ket{0}) \otimes (H\ket{0}) \xrightarrow{\text{}}\ket{+} \ket{+} \hspace{10mm}(\hspace{2mm}From \hspace{2mm}Hint\hspace{2mm})\]
\\
For $\ket{-}\ket{+}\xrightarrow{\text{CNOT}}\ket{-}\ket{+}$
\[\ket{\psi_{1}} = (H\ket{-}) \otimes (H\ket{+}) \xrightarrow{\text{}}\ket{1} \ket{0} \hspace{10mm}(\hspace{2mm}From \hspace{2mm}Hint\hspace{2mm})\]
\[\hspace{5mm}\ket{\psi_{2}} =(CNOT-INV)(\ket{\psi_{1}}) \xrightarrow{\text{}} {\text{}}\ket{1}\ket{0} \hspace{5mm}(\hspace{2mm}Target \hspace{2mm}Qubit \hspace{2mm} is\hspace{2mm}0\hspace{2mm})\]
\[\ket{\psi_{3}} = (H\ket{1}) \otimes (H\ket{0}) \xrightarrow{\text{}}\ket{-} \ket{+} \hspace{10mm}(\hspace{2mm}From \hspace{2mm}Hint\hspace{2mm})\]

For $\ket{+}\ket{-}\xrightarrow{\text{CNOT}}\ket{-}\ket{-}$
\[\ket{\psi_{1}} = (H\ket{+}) \otimes (H\ket{-}) \xrightarrow{\text{}}\ket{0} \ket{1} \hspace{10mm}(\hspace{2mm}From \hspace{2mm}Hint\hspace{2mm})\]
\[\hspace{5mm}\ket{\psi_{2}} =(CNOT-INV)(\ket{\psi_{1}}) = (CNOT-INV\ket{0}) \otimes \ket{1}\xrightarrow{\text{}} {\text{}}\ket{1}\ket{1} \hspace{5mm}(\hspace{2mm}Target \hspace{2mm}Qubit \hspace{2mm} is\hspace{2mm}1\hspace{2mm})\]
\[\ket{\psi_{3}} = (H\ket{1}) \otimes (H\ket{1}) \xrightarrow{\text{}}\ket{-} \ket{-} \hspace{10mm}(\hspace{2mm}From \hspace{2mm}Hint\hspace{2mm})\]

For $\ket{-}\ket{-}\xrightarrow{\text{CNOT}}\ket{+}\ket{-}$
\[\ket{\psi_{1}} = (H\ket{-}) \otimes (H\ket{-}) \xrightarrow{\text{}}\ket{1} \ket{1} \hspace{10mm}(\hspace{2mm}From \hspace{2mm}Hint\hspace{2mm})\]
\[\hspace{5mm}\ket{\psi_{2}} =(CNOT-INV)(\ket{\psi_{1}}) = (CNOT-INV\ket{1}) \otimes \ket{1}\xrightarrow{\text{}} {\text{}}\ket{0}\ket{1} \hspace{5mm}(\hspace{2mm}Target \hspace{2mm}Qubit \hspace{2mm} is\hspace{2mm}1\hspace{2mm})\]
\[\ket{\psi_{3}} = (H\ket{0}) \otimes (H\ket{1}) \xrightarrow{\text{}}\ket{+} \ket{-} \hspace{10mm}(\hspace{2mm}From \hspace{2mm}Hint\hspace{2mm})\]
\end{gather*}

\section*{Ex 3.2 The Stern-Gerlach Experiment}
    \section*{\small(a)}
    \begin{flushleft}
        For a square matrix \textbf{A} the Eigen Values and Eigen Vector satisfy: \[Av = \lambda v \hspace{25mm} \textbf{(1)}\]
    \linebreak\hspace{10mm}Where  \textbf{v} represents the Eigen Vectors and $\boldsymbol{\lambda}$ represents the Eigen Values.
    \linebreak \\
    Assuming there exists a non zero Eigen Vector we first solve for $\boldsymbol{\lambda}$ with the determinant \[|A - \lambda I| = 0 \hspace{17mm} \textbf{(2)}\]
    For Pauli-X Matrix :

    \[\bigg|\begin{pmatrix}
        0 & 1  \\
        1 & 0  \\
    \end{pmatrix} - \lambda \begin{pmatrix}
        1 & 0 \\
        0 & 1 \\
    \end{pmatrix} \bigg| = 0 \]

    \[\bigg|\begin{pmatrix}
        -\lambda & 1  \\
        1 & -\lambda  \\
    \end{pmatrix} \bigg| = 0 \]
    \[\lambda^{2} - 1 = 0 \implies \lambda = \pm 1\]
    Substitute $\lambda$ = 1 in \textbf{(1)}
    \[\begin{pmatrix}
        0 & 1  \\
        1 & 0  \\
    \end{pmatrix} \begin{pmatrix}
        x  \\
        y  \\
    \end{pmatrix} = \begin{pmatrix}
        x  \\
        y  \\
    \end{pmatrix}\]
    \begin{center}
        $\implies$ The Eigen Vector is a non-zero multiple of \emph{y=x}\\
    $\implies \begin{pmatrix}
        1 \\
        1
    \end{pmatrix}$ is a possible eigen vector for $\lambda = +1$
    \end{center}
    Similarly, for $\lambda = -1$
    \[\begin{pmatrix}
        0 & 1  \\
        1 & 0  \\
    \end{pmatrix} \begin{pmatrix}
        x  \\
        y  \\
    \end{pmatrix} = -\begin{pmatrix}
        x  \\
        y  \\
    \end{pmatrix}\]
    \begin{center}
        $\implies$ The Eigen Vector is a non-zero multiple of \emph{y= - x}\\
    $\implies \begin{pmatrix}
        1 \\
        -1
    \end{pmatrix}$ is a possible eigen vector for $\lambda = -1$
\end{center}
$\therefore$ Resultant normalized Eigen Vectors for Pauli-X Matrix are
\[\frac{1}{\sqrt{2}}\begin{pmatrix}
    1 \\
    1
\end{pmatrix}\hspace{5mm} \& \hspace{5mm}
\frac{1}{\sqrt{2}}\begin{pmatrix}
    1 \\
    -1
\end{pmatrix}\]
\\
Similarly, for Pauli-Y Matrix:
\[\bigg|\begin{pmatrix}
    0 & -i  \\
    i & 0  \\
\end{pmatrix} - \lambda \begin{pmatrix}
    1 & 0 \\
    0 & 1 \\
\end{pmatrix} \bigg| = 0 \]
\[\implies \lambda = \pm 1\]
Substituting $\lambda = \pm 1$ in \textbf{(1)} $\implies$ normalized Eigen Vectors for Pauli-Y Matrix are \[\frac{1}{\sqrt{2}}\begin{pmatrix}
    1 \\
    i \\
\end{pmatrix}\hspace{5mm} \& \hspace{5mm}
\frac{1}{\sqrt{2}}\begin{pmatrix}
    1 \\
    -i \\
\end{pmatrix}\]

Similarly, for Pauli-Z Matrix:
\[\bigg|\begin{pmatrix}
    1 & 0  \\
    0 & -1  \\
\end{pmatrix} - \lambda \begin{pmatrix}
    1 & 0 \\
    0 & 1 \\
\end{pmatrix} \bigg| = 0 \]
\[\implies \lambda = \pm 1\]
Substituting $\lambda = \pm 1$ in \textbf{(1)} $\implies$ normalized Eigen Vectors for Pauli-Z Matrix are \[\begin{pmatrix}
    1 \\
    0 \\
\end{pmatrix}\hspace{5mm} \& \hspace{4mm}
\begin{pmatrix}
    0 \\
    1 \\
\end{pmatrix}\]
\end{flushleft}

\section*{\small(b)}
\begin{flushleft}
    $\ket{\psi}$ = $\ket{0}$ (Given)\\~\\
Using Pauli-X Eigen Vectors as orthonormal basis\[u_{1} = \frac{1}{\sqrt{2}}\begin{pmatrix}
    1 \\
    1 \\
\end{pmatrix}=\ket{+}\hspace{5mm} \& \hspace{5mm}u_{2}=\frac{1}{\sqrt{2}}\begin{pmatrix}
    1 \\
    -1 \\
\end{pmatrix}=\ket{-}\] \\
Calculating Measurement Operator w.r.t computational basis \{$+,-$\}
% \underbrace{x}_\text{real} 
\[M_{+} := \ket{+}\bra{+} =\frac{1}{\sqrt{2}}\begin{pmatrix}
    1 \\
    1 \\
\end{pmatrix} \frac{1}{\sqrt{2}}\begin{pmatrix}
    1 & 1\\
    
\end{pmatrix} = \frac{1}{2}\begin{pmatrix}
    1 & 1\\
    1 & 1\\
\end{pmatrix}\]
\[M_{-} := \ket{-}\bra{-} =\frac{1}{\sqrt{2}}\begin{pmatrix}
    1 \\
    -1 \\
\end{pmatrix} \frac{1}{\sqrt{2}}\begin{pmatrix}
    1 & -1\\
    
\end{pmatrix} = \frac{1}{2}\begin{pmatrix}
    1 & -1\\
    -1 & 1\\
\end{pmatrix}\]
\[p(+)=\bra{0}M_{+}^{\dagger}M_{+}\ket{0} = \begin{pmatrix}
    1 & 0\\
\end{pmatrix} \frac{1}{2} \begin{pmatrix}
    1 & 1 \\
    1 & 1 \\
\end{pmatrix} \begin{pmatrix}
    1 \\
    0 \\
\end{pmatrix} = \frac{1}{2} = 50\%\]

\[p(-)=\bra{0}M_{-}^{\dagger}M_{-}\ket{0} = \begin{pmatrix}
    1 & 0\\
\end{pmatrix} \frac{1}{2} \begin{pmatrix}
    1 & -1 \\
    -1 & 1 \\
\end{pmatrix} \begin{pmatrix}
    1 \\
    0 \\
\end{pmatrix} = \frac{1}{2} = 50\%\] \\~\\
State after measurement corresponds to \[\frac{M_{m} \ket{\psi}}{\|M_{m} \ket{\psi}\|}\]
\[\implies \frac{M_{+} \ket{0}}{\|M_{+} \ket{0}\|} = \frac{\frac{1}{2} \begin{pmatrix}
    1 & 1\\
    1 & 1\\
\end{pmatrix} \begin{pmatrix}
    1 \\
    0 \\
\end{pmatrix}}{\Big\|\frac{1}{2} \begin{pmatrix}
    1 & 1\\
    1 & 1\\
\end{pmatrix} \begin{pmatrix}
    1 \\
    0 \\
\end{pmatrix}\Big\|}  = \frac{\frac{1}{2}\begin{pmatrix}
    1 \\
    1\\
\end{pmatrix}}{\frac{1}{\sqrt{2}}} = \frac{1}{\sqrt{2}}\Big( \ket{0} +\ket{1} \Big) = \ket{+}\]

\[\implies \frac{M_{-} \ket{0}}{\|M_{-} \ket{0}\|} = \frac{\frac{1}{2} \begin{pmatrix}
    1 & -1\\
    -1 & 1\\
\end{pmatrix} \begin{pmatrix}
    1 \\
    0 \\
\end{pmatrix}}{\Big\|\frac{1}{2} \begin{pmatrix}
     1 & -1 \\
    -1 &  1 \\
\end{pmatrix} \begin{pmatrix}
    1 \\
    0 \\
\end{pmatrix}\Big\|}  = \frac{\frac{1}{2}\begin{pmatrix}
    1 \\
    -1\\
\end{pmatrix}}{\frac{1}{\sqrt{2}}} = \frac{1}{\sqrt{2}}\Big( \ket{0} -\ket{1} \Big)=\ket{-}\]\\~\\ 

\begin{center}$\therefore$ The computed probabilities and output states are consistent w.r.t. the outputs of the non-uniform magnet applied in the X-direction \end{center}

\end{flushleft}

\section*{\small(c)}
\begin{flushleft}

 \underline{In Schematic Diagram 3}: When electrons in $\ket{+}$ superposition-state are passed through a non-uniform magnetic field along the Z-direction it is observed that 50\% of the electrons deflect up onto the metal-screen, i.e.; demonstrating a positive-spin ($\ket{0}$), while 50\% deflect downward, demonstrating a negative spin ($\ket{1}$). \\~\\
 These results can be verified numerically as follows: \\~\\
 $\ket{\psi}$ = $\ket{+}$ (Given)\\~\\
Using Pauli-Z Eigen Vectors as orthonormal basis\[u_{1} = \begin{pmatrix}
    1 \\
    0 \\
\end{pmatrix}=\ket{0}\hspace{5mm} \& \hspace{5mm}u_{2}=\begin{pmatrix}
    0 \\
    1 \\
\end{pmatrix}=\ket{1}\] \\
Calculating Measurement Operator w.r.t computational basis \{$u_{1},u_{2}$\}
% \underbrace{x}_\text{real} 
\[M_{0} := \ket{0}\bra{0} =\begin{pmatrix}
    1 \\
    0 \\
\end{pmatrix}\begin{pmatrix}
    1 & 0\\
    
\end{pmatrix} = \begin{pmatrix}
    1 & 0\\
    0 & 0\\
\end{pmatrix}\]
\[M_{1} := \ket{1}\bra{1} =\begin{pmatrix}
    0 \\
    1 \\
\end{pmatrix} \begin{pmatrix}
    0 & 1\\
    
\end{pmatrix} = \begin{pmatrix}
    0 & 0\\
    0 & 1\\
\end{pmatrix}\]

\[p(0)=\bra{+}M_{0}^{\dagger}M_{0}\ket{+} = \frac{1}{\sqrt{2}}\begin{pmatrix}
    1 & 1\\
\end{pmatrix} \begin{pmatrix}
    1 & 0 \\
    0 & 0 \\
\end{pmatrix} \frac{1}{\sqrt{2}} \begin{pmatrix}
    1 \\
    1 \\
\end{pmatrix} = \frac{1}{2} = 50\%\]

\[p(1)=\bra{+}M_{1}^{\dagger}M_{1}\ket{+} = \frac{1}{\sqrt{2}}\begin{pmatrix}
    1 & 1\\
\end{pmatrix} \begin{pmatrix}
    0 & 0 \\
    0 & 1 \\
\end{pmatrix} \frac{1}{\sqrt{2}}\begin{pmatrix}
    1 \\
    1 \\
\end{pmatrix} = \frac{1}{2} = 50\%\]

State after measurement corresponds to \[\frac{M_{m} \ket{\psi}}{\|M_{m} \ket{\psi}\|}\]
\[\implies \frac{M_{0} \ket{+}}{\|M_{0} \ket{+}\|} = \frac{ \begin{pmatrix}
    1 & 0\\
    0 & 0\\
\end{pmatrix} \frac{1}{\sqrt{2}}\begin{pmatrix}
    1 \\
    1 \\
\end{pmatrix}}{\Big\| \begin{pmatrix}
    1 & 0\\
    0 & 0\\
\end{pmatrix} \frac{1}{\sqrt{2}}\begin{pmatrix}
    1 \\
    1 \\
\end{pmatrix}\Big\|}  = \begin{pmatrix}
    1 \\
    0\\
\end{pmatrix} =\ket{0}\]

\[\implies \frac{M_{1} \ket{+}}{\|M_{1} \ket{+}\|} = \frac{ \begin{pmatrix}
    0 & 0\\
    0 & 1\\
\end{pmatrix} \frac{1}{\sqrt{2}}\begin{pmatrix}
    1 \\
    1 \\
\end{pmatrix}}{\Big\| \begin{pmatrix}
    0 & 0\\
    0 & 1\\
\end{pmatrix} \frac{1}{\sqrt{2}}\begin{pmatrix}
    1 \\
    1 \\
\end{pmatrix}\Big\|}  = \begin{pmatrix}
    0 \\
    1 \\
\end{pmatrix} =\ket{1}\]\\~\\

\underline{In Schematic Diagram 3}: Orienting the last magnetic field along the x-direction from z-direction.
$\ket{\psi}$ = $\ket{+}$ (Given)\\~\\

\[M_{+} = \frac{1}{2}\begin{pmatrix}
    1 & 1\\
    1 & 1\\
\end{pmatrix} \hspace{5mm} \& \hspace{5mm} M_{-}=\frac{1}{2}\begin{pmatrix}
    1 & -1\\
    -1 & 1\\
\end{pmatrix}\]

\[p(+)=\bra{+}M_{+}^{\dagger}M_{+}\ket{+} = \frac{1}{\sqrt{2}}\begin{pmatrix}
    1 & 1\\
\end{pmatrix} \frac{1}{2}\begin{pmatrix}
    1 & 1 \\
    1 & 1 \\
\end{pmatrix} \frac{1}{\sqrt{2}} \begin{pmatrix}
    1 \\
    1 \\
\end{pmatrix} = 1 = 100\%\]

\[p(-)=\bra{+}M_{-}^{\dagger}M_{-}\ket{+} = \frac{1}{\sqrt{2}}\begin{pmatrix}
    1 & 1\\
\end{pmatrix} \frac{1}{2}\begin{pmatrix}
    1 & -1 \\
    -1 & 1 \\
\end{pmatrix} \frac{1}{\sqrt{2}} \begin{pmatrix}
    1 \\
    1 \\
\end{pmatrix} = 0\%\]

State after measurement corresponds to
\[\implies \frac{M_{+} \ket{+}}{\|M_{+} \ket{+}\|} = \frac{ \frac{1}{2}\begin{pmatrix}
    1 & 1\\
    1 & 1\\
\end{pmatrix} \frac{1}{\sqrt{2}}\begin{pmatrix}
    1 \\
    1 \\
\end{pmatrix}}{\Big\| \frac{1}{2}\begin{pmatrix}
    1 & 1\\
    1 & 1\\
\end{pmatrix} \frac{1}{\sqrt{2}}\begin{pmatrix}
    1 \\
    1 \\
\end{pmatrix}\Big\|}  = \frac{1}{\sqrt{2}}\begin{pmatrix}
    1 \\
    1 \\
\end{pmatrix} =\ket{+}\]
\end{flushleft}

\end{document}