
\documentclass[a4paper,12pt]{article}
\usepackage{enumitem} % -> Alphabetical Lists
\usepackage{amsmath} % -> Matrices
\usepackage{fullpage} % -> A4 Full Page
\usepackage{amssymb} % -> Therefore
\usepackage[utf8]{inputenc}
\usepackage{graphicx}
\usepackage{adjustbox}
\usepackage{listings}
\usepackage{braket}
\usepackage{geometry}
\usepackage{tikz}
\usepackage{gensymb}

\usetikzlibrary{quantikz}

\graphicspath{ {./} }

\geometry{
    a4paper,
    total={175mm,257mm},
    left=20mm,
    top=20mm,
}

\title{Quantum Computing Assignment 8 - Group 18}
\author{
    Rallabhandi, Anand Krishna 
    \and
    Mustafa, Syed Husain
    \and
     , Mohammed Kamran 
}
\date{\today}

\begin{document}

\maketitle

\section*{Exercise 8.1}

 \begin{large}(\textbf{Decomposition of Controlled-U Gates})
 \end{large}

\begin{enumerate}[label=(\alph*)]
\item Consider $X = \begin{pmatrix}
    0 & 1\\
    1 & 0 \\
\end{pmatrix}$, $Y=\begin{pmatrix}
    0 & -i \\
    i & 0 \\
\end{pmatrix}$, \& $Z = \begin{pmatrix}
    1 & \phantom{-}0 \\
    0 & -1
\end{pmatrix}$\\~\\
\textbf{XYX} = $\begin{pmatrix}
    0 & 1 \\
    1 & 0 \\
\end{pmatrix}\begin{pmatrix}
    0 & -i \\
    i & \phantom{-}0 \\
\end{pmatrix}\begin{pmatrix}
    0 & 1 \\
    1 & 0 \\
\end{pmatrix} = \begin{pmatrix}
0 & 1\\
1 & 0 \\
\end{pmatrix} \begin{pmatrix}
    -i & 0 \\
    \phantom{-}0 & i \\
\end{pmatrix}= \begin{pmatrix}
    \phantom{-}0 & i \\
    -i & 0 \\
\end{pmatrix} = \textbf{-Y}$\\~\\
\textbf{XZX} = $\begin{pmatrix}
    0 & 1 \\
    1 & 0 \\
\end{pmatrix}\begin{pmatrix}
    1 & \phantom{-}0 \\
    0 & -1 \\
\end{pmatrix}\begin{pmatrix}
    0 & 1 \\
    1 & 0 \\
\end{pmatrix} = \begin{pmatrix}
    0 & 1\\
    1 & 0 \\
\end{pmatrix}\begin{pmatrix}
    \phantom{-}0 & 1 \\
    -1 & 0 \\
\end{pmatrix} = \begin{pmatrix}
    -1 & 0 \\
    \phantom{-}0 & 1\\
\end{pmatrix}= \textbf{-Z}$\\~\\
\textbf{X$\boldsymbol{R_y(\theta)}$X} = $\begin{pmatrix}
    0 & 1 \\
    1 & 0 \\
\end{pmatrix}\begin{pmatrix}
    \cos(\frac{\theta}{2}) & -\sin(\frac{\theta}{2}) \\
    \sin(\frac{\theta}{2}) & \cos(\frac{\theta}{2}) \\
\end{pmatrix}\begin{pmatrix}
    0 & 1 \\
    1 & 0 \\
\end{pmatrix} = \begin{pmatrix}
    \cos(\frac{\theta}{2}) & \sin(\frac{\theta}{2}) \\
    -\sin(\frac{\theta}{2}) & \cos(\frac{\theta}{2})\\
\end{pmatrix} = \boldsymbol{R_y(-\theta)}$ \\~\\
\underline{Properties Used} : $\cos(\phi) = \cos(-\phi)$ \& $-\sin(\phi) = \sin(-\phi)$\\~\\
\textbf{X$\boldsymbol{R_z(\theta)}$X} = $\begin{pmatrix}
    0 & 1 \\
    1 & 0 \\
\end{pmatrix}\begin{pmatrix}
    e^{\frac{-i\theta}{2}} & 0 \\
    0 & e^{\frac{i\theta}{2}} \\
\end{pmatrix}\begin{pmatrix}
    0 & 1 \\
    1 & 0 \\
\end{pmatrix} = \begin{pmatrix}
    0 & 1 \\
    1 & 0 \\
\end{pmatrix}\begin{pmatrix}
    0 & e^{\frac{-i\theta}{2}} \\
    e^{\frac{i\theta}{2}} & 0 \\
\end{pmatrix}= \begin{pmatrix}
    e^{\frac{i\theta}{2}} & 0 \\
    0 & e^{\frac{-i\theta}{2}} \\
\end{pmatrix} = \boldsymbol{R_z(-\theta)}$ \\~\\
\underline{Properties Used} : $e^{i\phi} = \cos(\phi) + i\sin(\phi)$ \& $e^{-i\phi} = \cos(\phi) - i\sin(\phi)$



\item \begin{gather*}
    ABC= R_z(\beta)*R_y(\frac{\gamma}{2})*R_y(\frac{-\gamma}{2})*R_z(\frac{-(\delta + \beta)}{2})*R_z(\frac{(\delta - \beta)}{2})
    \\~\\
    R_y(\frac{\gamma}{2})*R_y(\frac{-\gamma}{2}) = I
    \\~\\
    ABC= \begin{pmatrix}
e^\frac{-i\beta}{2} & 0 \\
0 & e^\frac{i\beta}{2} 
\end{pmatrix} * \begin{pmatrix}
e^\frac{i(\delta + \beta)}{4} & 0 \\
0 & e^\frac{-i(\delta + \beta)}{4} 
\end{pmatrix}* \begin{pmatrix}
e^\frac{-i(\delta - \beta)}{4} & 0 \\
0 & e^\frac{i(\delta - \beta)}{4} 
\end{pmatrix}\\~\\
= \begin{pmatrix}
e^\frac{i(\delta - \beta)}{4} & 0 \\
0 & e^\frac{-i(\delta - \beta)}{4} 
\end{pmatrix}* \begin{pmatrix}
e^\frac{-i(\delta - \beta)}{4} & 0 \\
0 & e^\frac{i(\delta - \beta)}{4} 
\end{pmatrix} = I
\end{gather*}\\~\\
To prove: $U=e^{i\alpha}AXBXC$\\~\\
Taking the right-hand side of the equation:\\~\\
\begin{gather*}
    e^{i\alpha}[R_z(\beta)*R_y(\frac{\gamma}{2})]* X * R_y(\frac{-\gamma}{2})*R_z(\frac{-(\delta + \beta)}{2}) * X * R_z(\frac{(\delta - \beta)}{2})\\ ~\\
    = e^{i\alpha}[R_z(\beta)*R_y(\frac{\gamma}{2})]* [X * R_y(\frac{-\gamma}{2})* X] * [ X * R_z(\frac{-(\delta + \beta)}{2}) * X] * R_z(\frac{(\delta - \beta)}{2})\\~\\
    = e^{i\alpha}[R_z(\beta)*R_y(\frac{\gamma}{2})] * R_y(\frac{\gamma}{2}) * R_z(\frac{(\delta + \beta)}{2}) *
    R_z(\frac{(\delta - \beta)}{2})\\~\\
    =e^{i\alpha}[R_z(\beta)*R_y(\frac{\gamma}{2})] * R_y(\frac{\gamma}{2}) *  \begin{pmatrix}
e^\frac{-i\delta}{2} & 0 \\
0 & e^\frac{i\delta}{2}
\end{pmatrix}\\~\\
=  e^{i\alpha}R_z(\beta) * \begin{pmatrix}
\cos{\frac{\gamma}{2}} & -\sin{\frac{\gamma}{2}} \\
\sin{\frac{\gamma}{2}} & \cos{\frac{\gamma}{2}}
\end{pmatrix}* \begin{pmatrix}
e^\frac{-i\delta}{2} & 0 \\
0 & e^\frac{i\delta}{2}
\end{pmatrix}\\~\\
=  e^{i\alpha}R_z(\beta)R_y(\gamma)R_z(\delta) =U
\end{gather*}
\item On solving the quantum circuit using matrix representation we get,
\\~\\
\begin{gather*}
\begin{pmatrix}
A & 0 \\
0 & Ae^{i\alpha}
\end{pmatrix} * \begin{pmatrix}
I & 0 \\
0 & X
\end{pmatrix} *\begin{pmatrix}
B & 0 \\
0 & B
\end{pmatrix} *\begin{pmatrix}
I & 0 \\
0 & X
\end{pmatrix} *\begin{pmatrix}
C & 0 \\
0 & C
\end{pmatrix}\\~\\
= \begin{pmatrix}
A & 0 \\
0 & AXe^{i\alpha}
\end{pmatrix}* \begin{pmatrix}
B & 0 \\
0 & B
\end{pmatrix} *\begin{pmatrix}
I & 0 \\
0 & X
\end{pmatrix} *\begin{pmatrix}
C & 0 \\
0 & C
\end{pmatrix}\\~\\
=
\begin{pmatrix}
AB & 0 \\
0 & AXBe^{i\alpha}
\end{pmatrix} *\begin{pmatrix}
I & 0 \\
0 & X
\end{pmatrix} *\begin{pmatrix}
C & 0 \\
0 & C
\end{pmatrix}\\~\\
=\begin{pmatrix}
AB & 0 \\
0 & AXBXe^{i\alpha}
\end{pmatrix} * \begin{pmatrix}
C & 0 \\
0 & C
\end{pmatrix}\\~\\
= \begin{pmatrix}
ABC & 0 \\
0 & e^{i\alpha}AXBXC
\end{pmatrix}
\\~\\
\begin{pmatrix}
I & 0 \\
0 & U
\end{pmatrix} = U-controlled
\end{gather*}
\item Any Unitary Operation can be decomposed as follows: \\
\[U = e^{i\alpha}R_{z}(\beta)R_{y}(\gamma)R_{z}(\delta) = \begin{pmatrix}
    e^{i(\alpha - \frac{\beta}{2}-\frac{\delta}{2})}\cos(\frac{\gamma}{2}) & -e^{i(\alpha - \frac{\beta}{2}+\frac{\delta}{2})}\sin(\frac{\gamma}{2}) \\
    e^{i(\alpha + \frac{\beta}{2}-\frac{\delta}{2})}\sin(\frac{\gamma}{2}) & e^{i(\alpha + \frac{\beta}{2}+\frac{\delta}{2})}\cos(\frac{\gamma}{2}) \\
\end{pmatrix}\]
Choosing $\beta=\gamma=0$\\
$\implies U = \begin{pmatrix}
    e^{i(\alpha - \frac{\delta}{2})} & 0 \\
    0 & e^{i(\alpha + \frac{\delta}{2})} \\
\end{pmatrix} \equiv \begin{pmatrix}
    1 & 0 \\
    0 & e^{\frac{2\pi i}{2^k}}
\end{pmatrix} \implies \alpha -\frac{\delta}{2}=0$  \& $\alpha + \frac{\delta}{2} = \frac{2\pi}{2^k} \implies \alpha = \frac{2\pi}{2^{k+1}} $ \& $\delta = \frac{2\pi}{2^k}$\\~\\
Suppose U is a Unitary Gate on a single qubit. Then there exists unitary operators A, B, \& C on a single qubit such that ABC=I \& $U=e^{i\alpha}AXBXC$, where $\alpha$ is some overall phase vector.\\
$A = R_z(\beta)R_y(\frac{\gamma}{2}),$ $B = R_y(\frac{-\gamma}{2})R_z(\frac{-(\delta+\beta)}{2}),$ \& $C=R_z(\frac{\delta-\beta}{2})$ \\ 
$\implies R_k = e^{\frac{2\pi i}{2^{k+1}}}R_z(0)R_y(0)XR_y(0)R_z(\frac{-2\pi}{2^{k+1}})XR_z(\frac{2\pi}{2^{k+1}}) = e^{\frac{2\pi i}{2^{k+1}}}XR_z(\frac{-2\pi}{2^{k+1}})XR_z(\frac{2\pi}{2^{k+1}})$
\[\begin{quantikz}
    \lstick{$\ket{q_0}$} & \ctrl{1} & \qw \\
    \lstick{$\ket{q_1}$} & \gate{R_k} &\qw  \\ 
\end{quantikz}\equiv\begin{quantikz}
    \lstick{$\ket{q_0}$} & \qw &\ctrl{1} & \qw& \ctrl{1}  & \gate{\begin{pmatrix}
        1 & 0 \\
        0 & e^{\frac{2 \pi i}{2^{k+1}}}
    \end{pmatrix}} & \qw \\
    \lstick{$\ket{q_1}$} & \gate{R_z(\frac{2\pi }{2^{k+1}})} &\targ{} & \gate{R_z(\frac{-2\pi }{2^{k+1}})} & \targ{} &\qw &\qw  \\
\end{quantikz}\]

\end{enumerate} \pagebreak
\section*{Exercise 8.2}

 \begin{large}(\textbf{Three Qubit Quantum Fourier Transform Implementation}) \\~\\
    \graphicspath{.}
    \includegraphics[scale=0.60]{Soln-8.2.png}\\~\\
   

    
 \end{large}
\end{document}
